%Example of use of oxmathproblems latex class for problem sheets
\documentclass{oxmathproblems}
\usepackage{algorithmic}
\usepackage{algorithm,float}
\usepackage{hyperref}
\hypersetup{hidelinks,
	colorlinks=true,
	allcolors=black,
	pdfstartview=Fit,
	breaklinks=true}

%(un)comment this line to enable/disable output of any solutions in the file
%\printanswers

%define the page header/title info
\course{Algorithm Design and Analysis (Fall 2023)}
\sheettitle{Assignment 1\\Deadline: Nov 1, 2023\\\small{Yuqing Tu 522030910152}} %can leave out if no title per sheet

\begin{document}
\begin{questions}

\makeatletter
\newenvironment{breakablealgorithm}
  {% \begin{breakablealgorithm}
   \begin{center}
     \refstepcounter{algorithm}% New algorithm
     \hrule height.8pt depth0pt \kern2pt% \@fs@pre for \@fs@ruled 画线
     \renewcommand{\caption}[2][\relax]{% Make a new \caption
       {\raggedright\textbf{\ALG@name~\thealgorithm} ##2\par}%
       \ifx\relax##1\relax % #1 is \relax
         \addcontentsline{loa}{algorithm}{\protect\numberline{\thealgorithm}##2}%
       \else % #1 is not \relax
         \addcontentsline{loa}{algorithm}{\protect\numberline{\thealgorithm}##1}%
       \fi
       \kern2pt\hrule\kern2pt
     }
  }{% \end{breakablealgorithm}
     \kern2pt\hrule\relax% \@fs@post for \@fs@ruled 画线
   \end{center}
  }
\makeatother

\miquestion[25]
Prove the following generalization of the master theorem. Given constants $a\geq 1,b> 1,d\geq 0$, and $w\geq 0$, if $T(n)=1$ for $n<b$ and $T(n) = aT(n/b) + n^d\log^w n$, we have
  $$
    T(n) = \begin{cases}
      O(n^d\log^w n) & \mbox{if }a < b^d \\
      O(n^{\log_b a}) & \mbox{if }a > b^d \\
      O(n^d\log^{w+1} n) & \mbox{if }a = b^d
    \end{cases}.
  $$

\textbf{Proof:}

The running time of solving all size-1 problem is
$$a^{\log_b n}O(1) = O(n^{\log_b a})$$
The total running time for combinig is
\begin{align}
  &\ \ \ \ c(n^d)(\log^w n) + ac(\frac{n}{b})^d(\log^w (\frac{n}{b})) + \ldots + a^kc(\frac{n}{b^k})^d(\log^w (\frac{n}{b^k})) + \ldots + a^{\log_b n}c(\log^w (\frac{n}{b^{\log_b n}})) \notag \\
  &<cn^d\log^w n(1 + (\frac{a}{b^d}) + \ldots + (\frac{a}{b^d})^k + \ldots + (\frac{a}{b^d})^{\log_b n}) \notag
\end{align}
\begin{parts}
\part $a < b^d:$
$$T(n) = O(n^d\log^w n)(1 + (\frac{a}{b^d}) + \ldots + (\frac{a}{b^d})^k + \ldots + (\frac{a}{b^d})^{\log_b n})$$
Since $a < b^d$, then $\frac{a}{b^d} < 1$
$$T(n) = O(n^d\log^w n)$$
\part $a > b^d:$ \\
$\because a > b^d$ \\
$\therefore \log_b a > d$ \\
$\therefore \exists \varepsilon > 0,\ s.t.\ n^d\log^w n \leq n^{\log_b a - \varepsilon}$ \\
$\therefore n^d\log^w n = O(n^{\log_b a - \varepsilon})$ \\
$\therefore $\ According to \href{https://blog.csdn.net/qq_41739364/article/details/101224786}{https://blog.csdn.net/qq\_41739364/article/details/101224786}
$$T(n) = O(n^{\log_b a})$$
\part $a = b^d:$
$$T(n) = O(n^d\log^w n)(1 + (\frac{a}{b^d}) + \ldots + (\frac{a}{b^d})^k + \ldots + (\frac{a}{b^d})^{\log_b n})$$
Since $a = b^d$, then $\frac{a}{b^d} = 1$
$$T(n) = O(n^d\log^{w+1} n)$$
\end{parts}

\miquestion[25]
Recall the median-of-the-medians algorithm we learned in the lecture. It groups the numbers by $5$. What happens if we group them by $3,7,9, \dots$? Please analyze those different choices and discuss which one is the best. 

\textbf{Solution:}

If we group them by $k$, we should find $\frac{n}{k}$ medians and find the median $v$ of them. Then we can draw a rectangle with length $\frac{n}{k}$ and width $k$. In this rectangle, there is a smaller rectangle with length $\frac{n}{2k}$ and width $\frac{k+1}{2}$, which contains all numbers greater or less than $v$. So at least $\frac{k+1}{4k}n$ numbers are definitely greater or less than $v$. Each recursion can eliminate $\frac{n}{3}$ numbers for $k = 3$, can eliminate $\frac{2n}{7}$ numbers for $k = 7$, can eliminate $\frac{5n}{18}$ numbers for $k = 9$.

When $k$ increases, the fewer numbers we can eliminate in each recursion and it leads to more recursive calls, but we spend less time to find the median-of-the-medians. 

To sum up, we should choose a suitable $k$ according to $n$, so that the total time between recursive calls and finding the median-of-the-medians is minimized.

\miquestion[25]
Let $X$ and $Y$ be two sets of integers.
Write $X\succ Y$ if $x\geq y$ for all $x\in X$ and $y\in Y$.
Given a set of $m$ integers, design an $O(m\log(m/n))$ time algorithm that partition these $m$ integers to $k$ groups $X_1,\ldots,X_k$ such that $X_i\succ X_j$ for any $i>j$ and $|X_1|,\ldots,|X_k|\leq n$.
Notice that $k$ is not specified as an input; you can decide the number of the groups in the partition, as long as the partition satisfies the given conditions.
You need to show that your algorithm runs in $O(m\log(m/n))$ time.

Remark: We have not formally define the asymptotic notation for multi-variable functions in the class.
For $f$ and $g$ be functions that maps $\mathbb{R}_{>0}^k$ to $\mathbb{R}_{>0}$, we say $f(\mathbf{x})=O(g(\mathbf{x}))$ if there exist constants $M,C>0$ such that $f(\mathbf{x})\leq C\cdot g(\mathbf{x})$ for all $\mathbf{x}$ with $x_i\geq M$ for some $i$.
The most rigorously running time should be written as $O(m\cdot\max\{\log(m/n),1\})$, although it is commonly just written as $O(m\log(m/n))$ for this kind of scenarios.

\textbf{Solution:}

Put these $m$ integers in array $A$, and execute the function $Partition(0, m-1, n)$.
\begin{breakablealgorithm}
\caption{$Partition(begin, end, n)$}
\begin{algorithmic}[1]
\IF {$m \leq n$}
\RETURN $A[begin, end]$
\ELSE
\STATE Use the median-of-the-medians algorithm to find the $\frac{begin+end}{2}$-th smallest integer $x^*$ in $A[begin, end]$
\STATE $Partition(begin, \frac{begin+end}{2}, n)$
\STATE $Partition(\frac{begin+end}{2}+1, end, n)$
\ENDIF
\end{algorithmic}
\end{breakablealgorithm}
Prove the algorithm runs in $O(m\log\frac{m}{n})$ time:
$$T(m) = 2T(\frac{m}{2}) + O(m)$$
Prove by induction that $T(m) \leq Am\log\frac{m}{n}$ for some constant $A$ \\
Since the median-of-the-medians algorithm runs in $O(n)$ times, \\
Base Step: $T(m) = 1$ for $m \leq n$ \\
Inductive Step: Suppose $T(i) \leq Ai\log\frac{i}{n}$ for $i > n$ and $i < m$ for $B-A\log2 \leq 0$
\begin{align}
  T(m) &\leq 2A\frac{m}{2}\log\frac{m}{2n} + Bm \notag \\
       &= Am\log\frac{m}{2n} + Bm \notag \\
       &= Am(\log\frac{m}{n} - \log2) + Bm \notag \\
       &= Am\log\frac{m}{n} + (B-A\log2)m \notag \\
       &\leq Am\log\frac{m}{n} \notag
\end{align}
So $T(n) = O(m\log\frac{m}{n})$.

\miquestion[25]
Given an array $A[1,\ldots,n]$ of integers sorted in ascending order, design an algorithm to \textbf{decide} if there exists an index $i$ such that $A[i]=i$ for each of the following scenarios. Your algorithm only needs to decide the existence of $i$; you do notGiven an array $A[1,\ldots,n]$ of integers sorted in ascending order, design an a need to find it if it exists.
\begin{parts}
\part The $n$ integers are positive and distinct.
\part The $n$ integers are distinct.
\part The $n$ integers are positive.
\part The $n$ integers are positive and are less than or equal to $n$.
\part No further information is known for the $n$ integers.
\end{parts}
Prove the correctness of your algorithms. For each part, try to design the algorithm with running time as low as possible.

\textbf{Solution:}

\begin{parts}
\part \begin{breakablealgorithm}
\caption{Decide the existence of $i$}
\begin{algorithmic}[1]
\IF {$A[1] = 1$}
\RETURN $True$
\ELSE
\RETURN $False$
\ENDIF
\end{algorithmic}
\end{breakablealgorithm}
\ \\
Prove by contradiction. \\
$\Longrightarrow :$ \\
Assume there is no $i$ such that $A[i]=i$ when $A[1] = 1$. Obviously there is a contradiction \\
$\Longleftarrow :$ \\
Assume $A[1] \neq 1$ when there is an $i$ such that $A[i]=i$. \\
$\because $ the $n$ integers are positive and $A[1] \neq 1$ \\
$\therefore $ $A[1] > 1$ \\
$\because $ $A[1, \ldots, n]$ is ascending and the $n$ integers are distinct \\
$\therefore $ $i - 1 \leq A[i] - A[1]$ \\
But there is an $i$ such that $A[i]=i$, then $i - 1 > A[i] - A[1]$. \\
There is a contradiction, so the algorithm is correct. \\
\ \\
\ \\
\ \\
\ \\
\part \begin{breakablealgorithm}
\caption{$deciding(begin, end)$}
\begin{algorithmic}[1]
\IF {$A[begin] > begin$ or $A[end] < end$}
\RETURN $False$
\ENDIF
\IF {$A[begin] = begin$ or $A[end] = end$}
\RETURN $True$
\ELSE
\RETURN $deciding(begin+1, \frac{begin+end}{2})\bigcup deciding(\frac{begin+end}{2}+1, end-1)$
\ENDIF
\end{algorithmic}
\end{breakablealgorithm}
\ \\
To prove this algorithm's correctness, we just need to prove if $A[begin] > begin$ or $A[end] < end$, then there is no $i$ for $begin < i < end$. \\
$\because A[begin] > begin$ and $A[begin+k] \geq A[begin]+k$ for $k = 1, 2, \cdots , end-begin$ \\
$\therefore A[begin+1] > begin+1$ \\
$\cdots \cdots \cdots $ \\
$\therefore A[end] > end$ \\
$\therefore $ There is no $i$ for $begin < i < end$. \\
Similarly, we can prove if $A[end] < end$, then there is no $i$ for $begin < i < end$.
So the algorithm is correct.
\part \begin{breakablealgorithm}
\caption{$deciding(begin, end)$}
\begin{algorithmic}[1]
\IF {$begin == end$ and $A[begin] \neq begin$}
\RETURN $False$
\ENDIF
\IF {$A[begin] > begin$ and $A[end] < end$}
\RETURN $True$
\ELSE
\RETURN $deciding(begin+1, \frac{begin+end}{2})\bigcup deciding(\frac{begin+end}{2}+1, end-1)$
\ENDIF
\end{algorithmic}
\end{breakablealgorithm}
\ \\
To prove this algorithm's correctness, we just need to prove if $A[begin] > begin$ and $A[end] < end$, then there is an $i$ for $begin < i < end$. \\
Prove by contradiction. \\
Assume $A[begin] > begin$ and $A[end] < end$ but there is no $i$ for $begin < i < end$. \\
Assume $A[begin] - begin = k$, so $A[begin+k] > A[begin] = begin + k$ \\
$\cdots \cdots \cdots $ \\
$\therefore A[end] > end$ \\
There is a contradiction, so the algorithm is correct. \\
\part It is always true. \\
Prove by contradiction. \\
Assume $0 < A[i] \leq n$ and there is no $i$ for $A[i] = i$. \\
$\therefore A[1] > 1$ \\
$\because A[2] \geq A[i]$ and $A[2] \neq 2$ \\
$\therefore A[2] > 2$ \\
$\cdots \cdots \cdots $ \\
$\therefore A[n] > n$ \\
There is a contradiction, so the algorithm is correct.
\part For every $i$, $i = 1, \ldots, n$: \\
If there is an $i$ that satisfies $A[i] = i$, then there exists an index $i$ such that $A[i]=i$. The correctness is obvious.
\end{parts}

\miquestion
How long does it take you to finish the assignment (including thinking and discussion)?
Give a score (1,2,3,4,5) to the difficulty.
Do you have any collaborators?
Please write down their names here.

It took me about ten hours to finish the assignment. I spent about four hours thinking and discussing, and the rest of the time getting familiar with the use of Latex and search for relevant syntax.
I'll give a 4 to the difficulty. My collaborator is Junqi You and his student ID is 522030910204.
\end{questions}
\end{document}
