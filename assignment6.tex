%Example of use of oxmathproblems latex class for problem sheets
\documentclass{oxmathproblems}
\usepackage{tikz}
\usepackage{algorithmic}
\usepackage{algorithm,float}
\usepackage{hyperref}
\hypersetup{hidelinks,
  colorlinks=true,
  allcolors=black,
  pdfstartview=Fit,
  breaklinks=true}
\usepackage{amsmath}  % 这个包提供了数学环境和命令

%(un)comment this line to enable/disable output of any solutions in the file
%\printanswers

%define the page header/title info
\course{Algorithm Design and Analysis (Fall 2023)}
\sheettitle{Assignment 6\\Deadline: Jan 9, 2023\\\small{Yuqing Tu 522030910152}} %can leave out if no title per sheet

\makeatletter
\newenvironment{breakablealgorithm}
  {% \begin{breakablealgorithm}
  \begin{center}
    \refstepcounter{algorithm}% New algorithm
    \hrule height.8pt depth0pt \kern2pt% \@fs@pre for \@fs@ruled 画线
    \renewcommand{\caption}[2][\relax]{% Make a new \caption
    {\raggedright\textbf{\ALG@name~\thealgorithm} ##2\par}%
    \ifx\relax##1\relax % #1 is \relax
      \addcontentsline{loa}{algorithm}{\protect\numberline{\thealgorithm}##2}%
    \else % #1 is not \relax
      \addcontentsline{loa}{algorithm}{\protect\numberline{\thealgorithm}##1}%
    \fi
    \kern2pt\hrule\kern2pt
    }
  }{% \end{breakablealgorithm}
    \kern2pt\hrule\relax% \@fs@post for \@fs@ruled 画线
  \end{center}
  }
\makeatother

\begin{document}
Choose \textbf{two} of the first four questions to submit.
Question 5 is the bonus question.

\begin{questions}


\miquestion
Prove that the following problem is NP-complete.
Given an undirected graph $G$ and an undirected graph $H$, decide if $H$ is a subgraph of $G$.

\miquestion
Prove that the following problem is NP-complete.
Given an undirected graph $G$ and a positive integer $k\geq 2$, decide if $G$ contains a spanning tree with maximum degree at most $k$.

\miquestion
Given an undirected graph $G=(V,E)$, prove that it is NP-complete to decide if $G$ contains an independent set with size \emph{exactly} $|V|/3$.

\textbf{Solution:}
\begin{enumerate}
  \item Prove $|V|/3$ IndependentSet is in NP:\\
  To verify whether a vertices set $S$ is an independent set with size $|V|/3$, we just need to verify whether $|S| = |V|/3$, whose time complexity is $O(|V|)$, and traverse $E$ to verify whether there are some edge $e_i \in E$ that $e_i$'s two endpoints are both in $S$, whose time complexity is $O(|E|)$ if we use a hash table to decide whether the endpoints is in $S$. So the total time complexity is $O(|V|+|E|)$. so $|V|/3$ IndependentSet is in NP.
  \item Prove $|V|/3$ IndependentSet is NP-hard:
  \begin{itemize}
    \item In 3-CNF formula, for each clause, construct a triangle where there vertices represent there literals. Connect two vertices if one represents the negation of the other. Then we get a graph $G$.\\
    This reduction's time complexity is $O(|V|+|E|)$ because we just put each vertex and edge into $G$.
    \item ``yes to yes'': If this 3SAT has a solution, then each clause has at least one literal which is \texttt{true}. For each triangle, put exactly one vertex representing a \texttt{true} literal into $S$. Then $S$ is an independent set of $G$ and $|S| = |V|/3$ because in each triangle, one and only one vertex is in $S$.
    \item ``no to no'': Prove by contradiction. Assume that this 3SAT has no solution, but there is a vertices set $S$ that is $G$'s independent set and $|S| = |V|/3$. There are $|V|/3$ triangles in $G$ totally, and in each triangle, one and only one vertex is in $S$. Otherwise there is at least one edge between some vertices in $S$. Then we assign \texttt{true} to all the literals representing the vertices in $S$. So each clause has one \texttt{true} literal, which means all clauses are \texttt{true}. Therefore, this 3SAT has a solution. There is a contradiction. ``no to no'' is true.
  \end{itemize}
  Because 3SAT is NP-complete and 3SAT $\leq_k$ $|V|/3$ IndependentSet, so $|V|/3$ IndependentSet is NP-hard.
\end{enumerate}
To sum up, $|V|/3$ IndependentSet is NP-complete.

\miquestion
Consider the decision version of \emph{Knapsack}. Given a set of $n$ items with weights $w_1,\ldots,w_n\in\mathbb{Z}^+$ and values $v_1,\ldots,v_n\in\mathbb{Z}^+$, a capacity constraint $C\in\mathbb{Z}^+$, and a positive integer $V\in\mathbb{Z}^+$, decide if there exists a subset of items with total weight at most $C$ and total value at least $V$. Prove that this decision version of Knapsack is NP-complete.

\textbf{Solution:}
\begin{enumerate}
  \item Prove Knapsack is in NP:\\
  To verify whether a integer set $S$ is a solution, we just need to compute $\sum_{i\in S}w_i$ and $\sum_{i\in S}v_i$. If $\sum_{i\in S}w_i \leq C$ and $\sum_{i\in S}v_i \geq V$, $S$ is a solution, else $S$ is not. And the time complexity is obviously $O(n)$. So Knapsack is in NP.
  \item Prove Knapsack is HP-hard:
  \begin{itemize}
    \item In a VectorSubsetSum problem, we construct $n$ 2-dimensional vectors $x_i = (v_i, w_i)$ for $i = 1, 2, \ldots, n$. We also construct enough 2-dimensional vectors $x_{n+2i-1} = (v_{n+2i-1}, w_{n+2i-1}) = (-1, 0)$ and $x_{n+2i} = (v_{n+2i}, w_{n+2i}) = (0, 1)$ for $i = 1, 2, \ldots$. Then we decide if there is a integer set $S$ making $\sum_{i\in S}x_i = (\sum_{i\in S}v_i, \sum_{i\in S}w_i) = (V, C)$.\\
    This reduction's time complexity is $O(n)$.
    \item ``yes to yes'': If there is a integer set $S$ making $\sum_{i\in S}x_i = (\sum_{i\in S}v_i, \sum_{i\in S}w_i) = (V, C)$, then we have $\sum_{i\in S, 1 \leq i \leq n}v_i \geq V$ and $\sum_{i\in S, 1 \leq i \leq n}w_i \leq C$, which is a solution of Knapsack.
    \item ``no to no'': Prove by contradiction. Assume that there is not a integer set $S'$ making $\sum_{i\in S'}x_i = (\sum_{i\in S'}v_i, \sum_{i\in S'}w_i) = (V, C)$, but Knapsack has a solution set $S$. Then we have $\sum_{i\in S}v_i = V' \geq V$ and $\sum_{i\in S}w_i = C' \leq C$. But
    \begin{align}
    &\ \ \ \ \sum_{i\in S}x_i + \sum_{1\leq i\leq V'-V}x_{n+2i-1} + \sum_{1\leq i\leq C-C'}x_{n+2i} \notag \\
    &=(\sum_{i\in S}v_i - (V'-V), \sum_{i\in S}w_i + (C-C')) \notag \\
    &=(V' - (V'-V), C' + (C-C')) \notag \\
    &=(V, C) \notag
    \end{align}
    So there is a set $S' = S + {n+2i-1, n+2j}$ for $i = 1, 2, \ldots, V'-V$ and $j = 1, 2, \ldots, C-C'$ making $\sum_{i\in S'}x_i = (\sum_{i\in S'}v_i, \sum_{i\in S'}w_i) = (V, C)$. There is a contradiction. ``no to no'' is true.
  \end{itemize}
  Because VectorSubsetSum is NP-complete and VectorSubsetSum $\leq_k$ Knapsack, so Knapsack is NP-hard.
\end{enumerate}
To sum up, the decision version of Knapsack is NP-complete.

\miquestion
(\textbf{Bonus})
In the class, we have seen that 3SAT is NP-complete.
In this question, we investigate the 2SAT problem and its variants.
Similar to the 3SAT problem, in the 2SAT problem, we are given a 2-CNF Boolean formula (where each clause contains two literals) and we are to decide if this formula is satisfiable.
\begin{parts}
\part Prove that 2SAT is in P. (Hint: a clause $(a_i\vee a_j)$ with two literals $a_i$ and $a_j$ can be represented as two logical implications: $\neg a_i\Longrightarrow a_j$ and $\neg a_j\Longrightarrow a_i$; you may want to construct a directed graph with $2n$ vertices corresponding to $x_1,\neg x_1,x_2,\neg x_2,\ldots,x_n,\neg x_n$.)
\part Consider this variant of the 2SAT problem: given a 2-CNF Boolean formula $\phi$ and a positive integer $k$, decide if there is a Boolean assignment to the variables such that at least $k$ clauses of $\phi$ are satisfied. Notice that 2SAT is the special case of this problem with $k$ equals to the number of the clauses. Prove that this problem is NP-complete.
\end{parts}

\textbf{Solution:}
\begin{parts}
  \part Construct $2n$ vertices representing to all literals and their negation in the 2-CNF formula $x_1,\neg x_1,x_2,\neg x_2,\ldots,x_n,\neg x_n$. And construct a directed edge from the vertex representing $\neg x_i$ to the vertex representing $x_j$ and a directed edge from the vertex representing $\neg x_j$ to the vertex representing $x_i$ if there is a clause $(x_i\vee x_j)$. Then we get a directed graph $G$. And this 2-CNF formula is not satisfiable iff there exists at least a pair of vertices representing $x_i$ and $\neg x_i$ that can reach to each other.

  Correctness proof:\\
  Because $\neg a_i\Longrightarrow a_j$ and $\neg a_j\Longrightarrow a_i$ if there is a clause $(a_i\vee a_j)$, so if we start from one vertex representing $x_i$ and assume $x_i$ is \texttt{true}, then all literals represented by the vertices we can reach to are \texttt{true}.
  \begin{itemize}
    \item ``if'': Because there exists at least a pair of vertices representing $x_i$ and $\neg x_i$ that can reach to each other, so $x_i$ and $\neg x_i$ can't be \texttt{true} and \texttt{false} separately. Then this 2-CNF formula mustn't be satisfiable.
    \item ``only if'': Because this 2-CNF formula is not satisfiable, so there is at least a pair literals $x_i$ and $\neg x_i$, if we want to make the formula \texttt{true}, $x_i$ and $\neg x_i$ must be \texttt{true} at a same time. Then the pair of vertices representing $x_i$ and $\neg x_i$ that can reach to each other.
  \end{itemize}

  Time complexity:\\
  We just need to run DFS from each vertices $x_i$ in $G$ and decide whether we can reach to $\neg x_i$, so the total time complexity is $O(|V|(|V|+|E|)) = O(n^3)$ and it is a polynomial time algorithm.

  So 2SAT is in P.
\end{parts}

\miquestion
How long does it take you to finish the assignment (including thinking and discussion)?
Give a score (1,2,3,4,5) to the difficulty.
Do you have any collaborators?
Please write down their names here.

It takes me about 2 hours to finish the assignment. I give 2 to the difficulty.

\end{questions}
\end{document}
